\documentclass[10pt,a4paper,landscape]{article}
\usepackage{amssymb,amsmath,amsthm,amsfonts}
\usepackage{multicol,multirow}
\usepackage{calc}
\usepackage{ifthen}
\usepackage[landscape]{geometry}
\usepackage[colorlinks=true,citecolor=blue,linkcolor=blue]{hyperref}
\usepackage{graphicx}
\usepackage{arydshln,bm,graphicx,hyperref,mathrsfs,enumerate,subfigure,float,graphics}




\ifthenelse{\lengthtest { \paperwidth = 11in}}
  { \geometry{top=.5in,left=.5in,right=.5in,bottom=.5in} }
	{\ifthenelse{ \lengthtest{ \paperwidth = 297mm}}
		{\geometry{top=1cm,left=.5cm,right=.5cm,bottom=1cm} }
		{\geometry{top=1cm,left=1cm,right=1cm,bottom=1cm} }
	}
\pagestyle{empty}
\makeatletter
\renewcommand{\section}{\@startsection{section}{1}{0mm}%
                {-1ex plus -.5ex minus -.2ex}%
                {0.5ex plus .2ex}%x
                {\normalfont\large\bfseries}}
\renewcommand{\subsection}{\@startsection{subsection}{2}{0mm}%
                {-1explus -.5ex minus -.2ex}%
                {0.5ex plus .2ex}%
                {\normalfont\normalsize\bfseries}}
\renewcommand{\subsubsection}{\@startsection{subsubsection}{3}{0mm}%
                {-1ex plus -.5ex minus -.2ex}%
                {1ex plus .2ex}%
                {\normalfont\small\bfseries}}
\makeatother
\setcounter{secnumdepth}{0}
\setlength{\parindent}{0pt}
\setlength{\parskip}{0pt plus 0.5ex}


\newcommand{\tabincell}[2]{\begin{tabular}{@{}#1@{}}#2\end{tabular}} 
\usepackage[labelfont=bf,labelsep=none]{caption}

\title{Sample Survey}

\begin{document}

\raggedright
\footnotesize

\begin{multicols}{3}

  \setlength{\premulticols}{1pt}
  \setlength{\postmulticols}{1pt}
  \setlength{\multicolsep}{1pt}
  \setlength{\columnsep}{2pt}



{\scriptsize
	
{\small\textbf{1 Basic Hypothesis Testing}}\\
\underline{Measures of variability}: standard deviation\& range:$x_{(n)}-x_{(1)}$ \& interquartile range(IQR):$Q_3-Q_1$, first, second, third quartiles.\\
\underline{Measures of shape}: positive/rights kewed, negative/left skewed skewness=$\frac{E[(X-\mu)^3]}{\sigma^3}$ (sample=$\frac{n}{(n-1)(n-2)}\sum_{i=1}^n \frac{(X_i-\overline{X})^3}{S^3}$)\\
kurtosis=$\frac{E[(X-\mu)^4]}{\sigma^4}-3$, measures the heaviness of tails, compared to a normal distribution. sample=$\frac{n(n+1)}{(n-1)(n-2)(n-3)}\sum_{i=1}^n\frac{(X_i-\overline{X})^4}{S^4}-\frac{3(n-1)^2}{(n-2)(n-3)} $.






{\small\textbf{3 Basic Hypothesis Testing}}\\
\textbf{3.1 Basic concepts}\\
Type I error($\alpha$): reject $H_0$ when $H_0$ is actually the truth. Type II error()$\beta$): fail to reject $H_0$ when $H_1$ is the truth.\\
$\alpha$=Pr($\boldsymbol{x}\in RR|H_0$ is true), $\beta$=Pr($\boldsymbol{x}\notin RR|H_1$ is true).\\
The p-value is the prob of obtaining test result at least as extreme as the result actually observed during the test, assuming $H_0$ is truth.\\
Due to the randomness of the observed data, p-value is r.v., which $\sim U[0,1]$\\
\underline{Statistic Power}: the prob of rejecting $H_0$ when $H_1$ is true./ power=1-$\beta$\\
\underline{Sampling dist}: dist of the point estimate based on samples of a fixed size from a population.\\
\underline{Interpretation of CI}: having numerous sample datasets and the 95\% CI is computed for each sample dataset, then the fraction of cmputed CI that encompass the true parameter would tend toward 95\%.

\textbf{3.2 Hypothesis Testing for Categorical Variables}\\
\textbf{One-sample z test}: $\hat{p}=\frac{\sum X_i}{n}$, SE of estimate $SE(\hat{p})=\sqrt{p(1-p)/n}$\\
test statistic: $Z=\frac{\hat{p}-p_0}{\sqrt{p_0(1-p_0)/n}} \sim_{asy} N(0,1)$.\\
Wald interval: $\hat{p}\pm z_{\alpha/2}\times \sqrt{\hat{p}(1-\hat{p})/n}$\\
Need for sample size:(CI)$n\hat{p}\geq 10$\&$n(1-\hat{p})\geq 10$ (test on p)$np_0\geq 10$\& $n(1-p_0)\geq10$ \\ 
 There are other types of intervals, e.g., the Wilson(or score)interval, the Clopper-Pearson(or exact)interval, etc\\
\textbf{Two-sample z test}: $H_0:p_1-p_2=0$, we have $\hat{p_1}=\frac{\sum X_{1i}}{n_1}$, $\hat{p_2}=\frac{\sum X_{2i}}{n_2}$\\
$SE(\hat{p_1}-\hat{p_2})=\sqrt{\frac{p_1(1-p_1)}{n_1}+\frac{p_2(1-p_2)}{n_2}}$, under $H_0$, $SE(\hat{p_1}-\hat{p_2})=\sqrt{p(1-p)(\frac{1}{n_1}+\frac{1}{n_2})}$\\
test statistic: $Z=\frac{(\hat{p}_1-\hat{p}_2)}{\sqrt{p(1-p)(\frac{1}{n_1}+\frac{1}{n_2})}}\sim _{asy} N(0,1)$ where p is estimated by pooled porportion $\hat{p}=\frac{n_1\hat{p}_1+n_2\hat{p}_2}{n_1+n_2}$\\
Need for sample size:(CI)$n_i\hat{p}_i\geq 10$\&$n_i(1-\hat{p}_i)\geq 10$ (test on p)$n_i \hat{p}\geq 10$\& $n_i(1-\hat{p})\geq10$ for i=1,2\\ 
\textbf{Test for contingency table: } Pearson's chi-square test  can be used to assess: Goodness of fit\&Homogeneity\&Independence\\
\underline{For Got}:$H_0: p_1=p_{01},...,p_m=p_{0m}$.
Pearson's chi-square test statistic is $\chi^2=\sum_{i=1}^m \frac{(O_i-E_i)^2}{E_i}\sim_{asy} \chi^2_{m-1} $, $O_i$ is the obs freq of the ith category, $E_i=np_{0i}$ is the expected freq under $H_0$.\\
\underline{For homog\&indep}:groups=r, category=c, $\chi^2=\sum_{i=1}^{r}\sum_{j=1}^{c}\frac{(O_{ij}-E_{ij})^2}{E_{ij}}$ $\sim_{asy} \chi^2_{(r-1)(c-1)}$ under hom or indep ass, where $E_{ij}$ is the expected freq of cell $(i,j)$ assuming indep: $E_{ij}=np_{i\cdot}p_{\cdot j}$, when r=c=2, $\chi^2$ equiv two sample z test for binary variables. \\
\underline{Ass for applay this test}:(1)obs $X_{1i}$ and $X_{2i}$ are indep samples (2)sample size is enough(cell counts greater or equal to 10)\\
When sample is small, apply exact tests to compute p-values: 1. Fisher's exact test: with the same margins(same row and col sums) 2.Barnard's exact test: only the row margins are fixed, more powerful than the Fisher's\\
\textbf{McNemar's Test for paired samples}: $H_0:p_1=p_2$

\textbf{3.3 Hypothesis testing for Continuous Variable}\\
\underline{One-sample t test}:(for normal population) $H_0: \mu=\mu_0$, test statistic $T=\frac{\sqrt{n}(\overline{X}-\mu_0)}{S}\sim t_{n-1}$(exact)\\
The larger the d.f., the more closely the dist approxmiates N(0,1)\\
By CLT, T asy follows N(0,1), under $H_0$, the t-test provides an exact test.\\
\underline{Two-sample t test}:(for 2 indep normal populations) $H_0:\mu_1=\mu_2$, if assuming that $\sigma_1=\sigma_2=\sigma$, then pooled sample standard deviation $S_p=\sqrt{\frac{1}{n_1+n_2-2}(\sum_{i=1}^{n_1}(X_{1i}-\overline{X}_1)^2+\sum_{i=1}^{n_2}(X_{2i}-\overline{X}_2)^2 )  }=\sqrt{\frac{(n_1-1)S_1^2+(n_2-1)S_2^2}{n_1+n_2-2} }$,
test statistic:$T=\frac{\overline{X}_1-\overline{X}_2}{\sqrt{S_p^2(\frac{1}{n_1}+\frac{1}{n_2} )}}\sim t_{n_1+n_2-2}$(exact)\\
\underline{F test}, $H_0: \sigma_1^2=\sigma_2^2$, test statistic: $F=\frac{S_1^2}{S_2^2}\sim F(n_1-1,n_2-1)$\\
when the variance are not equal, $SE(\overline{X}_1-\overline{X}_2)$ is better estimated by $\sqrt{S_1^2/n_1+S_2^2/n_2}$, thus the test statistic is
$T_s=\frac{\overline{X}_1-\overline{X}_2}{\sqrt{S_1^2/n_1+S_2^2/n_2}}\sim_{asy} t_{v} $, where $v=\frac{(S_1^2/n_1+S_2^2/n_2)^2}{ \frac{(S_1^2/n_1)^2 }{n_1-1} +\frac{(S_2^2/n_2)^2 }{n_2-1} } $ is called Satterthwaite/Welch's t-test.\\
\underline{Nonparametric test for Means/Median}: \textbf{Sign test:} $H_0:m=m_0$, let $N^+$ be the number of positive signs obtained upon calculating $X_i-m_0$ for $i=1,..,n$,under $H_0$, $N^+\sim Bin(n,p)$ with $p=0.5$, take one-sample z-test. (robust)\\
\textbf{Wilcoxon signed-rank test}: compute $\{X_i-m_0 \}_{i=1}^n$$\rightarrow$order $\{|X_i-m_0|\}_{i=1}^n$ and assign ranks$\rightarrow$sums of ranks(positive)$=S^+$\\
When sample size n is large$(>20)$, by CLT, under $H_0$, we have $W=S^+\sim_{asy} N(\frac{n(n+1)}{4},\frac{n(n+1)(2n+1)}{24})$\\
\textbf{3.4 Multiple Comparsions}\\
Type I error/alpha inflation. To control the family-wise error rate(FWER), i.e., the prob of incorrectly rejecting at least one $H_0$.\\
\textbf{Bonmferroni:}$\tilde{p}_i$=min\{$m\times p_i,1$\} or $\tilde{\alpha}_i=\alpha/m$ is conservative when m is large or the tests are highly positively correlated.\\
\textbf{Holm adjustment:} step 1:if $p_{(1)}\leq \alpha/m$, reject $H_{(1)0}$ and continue, else stop$\cdots$ step m:if $p_{(m)}\leq \alpha$, rejcet $H_{(m)0}$. with larger threshold (more powerful):$\tilde{\alpha}_i=\alpha/(m-i+1)$ \& adjusted p-value: $\tilde{p}_{(i)}= \{1, max\{ (m-i+1)p_{(i)},\tilde{p}_{(i-1)}  \}\}$








	
	
	

{\small\textbf{4 Linear Regression: Model Fitting}}\\
\textbf{4.1 The Multiple Linear Regression}\\
\underline{Regression analysis}: describe the mean of the distribution of one variable (response) as a function of other variables (explanatory):$E(Y|X)=f(X)$.\\
Regression Model:$\boldsymbol{y}=\boldsymbol{X\beta}+\boldsymbol{\epsilon}$: $\boldsymbol{X}$, design matrix, $\boldsymbol{\beta}$:vector of parameter.\\
\textbf{Least Square Method}: assumptions: 1.All explanatory variables $X_i$ are fixed 2.random errors are uncorrelated with $E(\epsilon)=0$\&$Var(\epsilon)=\sigma^2$.\\
Two results: $\frac{\partial{\boldsymbol{a}^T\boldsymbol{x}}}{\partial{\boldsymbol{x}}}=\frac{\partial{\boldsymbol{x}^T\boldsymbol{a}}}{\partial{\boldsymbol{x}}}=\boldsymbol{a}$ and $\frac{\partial{(x^TAx)}}{\partial{\boldsymbol{x}}}=(A+A^T)\boldsymbol{x}$\\
$SSE(\boldsymbol{\beta})=||\boldsymbol{y}-\boldsymbol{X}\boldsymbol{\beta}||^2$$\rightarrow$$\hat{\beta}=(\boldsymbol{X}^T\boldsymbol{X})^{-1}\boldsymbol{X}^T\boldsymbol{y}$$\rightarrow$fitted value(orthogonal projection):$\hat{\boldsymbol{y}}=\boldsymbol{X}(\boldsymbol{X}^T\boldsymbol{X})^{-1}\boldsymbol{X}^T\boldsymbol{y}=\boldsymbol{H}\boldsymbol{y}$,$\boldsymbol{H}$:hat matrix(projection matrix)\\
$E(\hat{\boldsymbol{\beta}})=\boldsymbol{\beta}, Var(\hat{\boldsymbol{\beta}})=\sigma^2(\boldsymbol{X X^T})^{-1}$, $\hat{\sigma}^2=\frac{RSS}{n-p-1}$ is a unbiased estimator.
\textbf{Maximum Likelihood Estimation}: Assumptions: 1. All explanatory variables $X_i$ are fixed 2.random errors are i.i.d. $N(0,\sigma^2)$\\
Under ass, have important results: 1. $\hat{\boldsymbol{\beta}}\sim N(\boldsymbol{\beta},\sigma^2(\boldsymbol{X}^T\boldsymbol{X})^{-1}) $ 2. $\frac{n\hat{\sigma}^2}{\sigma^2}=\frac{RSS}{\sigma^2}\sim\chi^2_{n-p-1}$ 3. $\hat{\boldsymbol{\beta}}$ and $\hat{\sigma}^2$ are independent.  need has two cons\\
If $\boldsymbol{A},\boldsymbol{B}$(scalars matrices) and $\boldsymbol{y}\sim N(\boldsymbol{\mu}, \boldsymbol{\Sigma})$, then 1. $\boldsymbol{Ay}\sim N(\boldsymbol{A\mu}, \boldsymbol{A\Sigma A^T})$ 2.$\boldsymbol{Ay}$ and $\boldsymbol{By}$ are indep iff $\boldsymbol{A\Sigma B}^T=0$


\textbf{4.2 Testing the Regression Coefficients}\\
\underline{Single}:$H_0:\beta_i=0$. Denote (i+1)-th diagonal element of $\boldsymbol{(X^TX)}^{-1}$ as $c_{ii}$, as $\hat{\boldsymbol{\beta}}\sim N(\boldsymbol{\beta},\sigma^2 (\boldsymbol{X^TX})^{-1})$, have $\hat{\beta_i}\sim N(\beta_i, c_{ii}\sigma^2)$, test statistic: $T=\frac{\hat{\beta}_i-0 }{c_{ii}\hat{\sigma }^2}\sim t_{n-p-1} $ under $H_0$\\
\underline{Several}:$H_0:\beta_{k+1}=\cdots=\beta_p=0$. Def two models, full model: $Y=\beta_0+\beta_1X_1+\cdots+\beta_kX_k+\cdots+\beta_pX_p+\epsilon$ and reduced/restricted model$(k<p)$: $Y=\beta_0+\beta_1X_1+\cdots+\beta_kX_k+\epsilon$$\rightarrow$$RSS_R\geq RSS_F$, test statistic: $F=\frac{(RSS_R-RSS_F)/(p-k)}{RSS_F/(n-p-1)} \sim F(p-k,n-p-1)$ under $H_0$\\
\underline{Overall significance},$H_0:\beta_1=\cdots=\beta_p=0$ $\rightarrow$ ANOVA table
\textbf{SST}:total sum of squares, \textbf{SSM}:explained sum of sqaures of the model, \textbf{SSE}: residual sum of squares\\
\textbf{R Squared/Coefficient of determination}: $R^2=SSM/SST$, represents the proportion of variance in the response variable that is explained by the explanatory variables, the remaining can be attributed to unknown variables or inherent variability.\\
\underline{Interactive effects}: two exp vars are said to interact if the effect that one of them has on the mean response depends on the value of the other.\\
\textbf{Gauss-Markov Theorem}: in linear regression model, if $\epsilon_1,\cdots,\epsilon_n$ satisfie:$E(\epsilon_i)=0$ and $Var(\epsilon_i)=\sigma^2<\infty$; $Cov(\epsilon_i,\epsilon_j)=0,\forall i\neq j$, then $\hat{\beta}_{LSE}$has the lowest sampling
variance within the class of linear unbiased estimators, termed the BLUE.\\




  {\small\textbf{5 Linear Regression: Model Selection and Diagnosis}}\\
  \textbf{5.1 Model Selection}\\
  With less variables: overfitting, simplicity/Interpretation \\
  Model-fitting criterion: $R^2_{adj}=1-\frac{MSE_k}{MST}=1-(1-R^2)\frac{n-1}{n-k-1} $, largest $R^2_{adj}$ is equiv to choose model smallest MSE\\
  Mallows's $C_p$: $C_p=\frac{SSE_k}{SSE_p/(n-p-1)}-(n-2k-2)$, under full model, k=p, $C_p=k+1=p+1$, can be proven that $E(C_p)=k+1$, choose the model with $C_p$ closest to k+1 and k is small.\\
  $AIC=-2log(\hat{L})+2(k+1)$, $BIC=-2log(\hat{L})+logn(k+1)$, when $n\geq 8$, BIC imposes heavier penalty on k than AIC, $l(\beta,\sigma^2)=-\frac{n}{2}log \sigma^2 -\frac{(y-X^T\beta)^T(y-X^T\beta)}{2\sigma^2} $+c, and $\hat{\sigma}^2=\frac{(y-X^T\beta)^T(y-X^T\beta)}{n}$ \\
  AIC=$nlog(\frac{SSE_k}{n})+2(k+1)+c$, BIC=$nlog(\frac{SSE_k}{n})+(k+1)log n+c$
  \underline{Sequential Selection}: Begin with the current model, sequentially add and/or drop one explanatory variable at a time based on whether the resulting model is superior.\\
  forward selection/ backward elimination/ stepwise selection\\
  \underline{Shrinkage method}: \\ 
  Ridge, minimize $SSE(\boldsymbol{\beta}, \lambda)=\sum_{i=1}^n(y_i-\beta_0-\cdots-\beta_px_{ip})^2+\lambda \sum_{j=1}^p \beta_j^2$ is equiv $SSE(\boldsymbol{\beta})=\sum_{i=1}^n(y_i-\beta_0-\cdots-\beta_px_{ip})^2$ subject $\lambda \sum_{j=1}^p \beta_j^2\leq t$\\
  $\hat{\boldsymbol{\beta}}^{ridge}=(\boldsymbol{X^TX}+\lambda \boldsymbol{I}_p)^{-1}\boldsymbol{X^Ty}$, addition of $\lambda \boldsymbol{I}_p$ makes nonsingular.\\
  Lasso: Because of the nature of the constraint, making  sufficiently small ( sufficiently large) will cause
  some of the coefficients to be exactly zero. \\
  \underline{Cross validation}: to test the model’s ability to predict new data to obtain an insight on how the model will generalize to an unknown dataset. \\
  Leave-one-out CV \& k-fold CV: Shuffle the data randomly and split the data into  groups of approximately equal size.\\
  
  \textbf{5.2 Model Diagnosis}\\
  Linearity\& Homoscedasticity\&Independence\&Normality, lin>hom>nor
  \underline{Residual plots}:$r_i=y_i-\hat{y}_i$, first check the linear and hom by Fitted(X) versus Residual Plot(Y), i.e., scatterplots of the residuals against the fitted.\\
  Noraml Qunantile-Q plot: check the normality ass, we expect that the points in the Q-Q plot will closely lie on a straight line, or histogram.
  
  \underline{Hypothesis}: \textbf{Hom}: \underline{Breusch-Pagan test} and \underline{White test}\\
  \underline{BP}: auxiliary regression moel: $r_i^2=\gamma_0+\gamma_1z_{i1}+\cdots+\gamma_k z_{ik}+e_i$, $H_0:\gamma_1=\cdots=\gamma_k=0$, using F-statistic.
  \underline{White}: All explan vars, all square vars,all intera terms are included. another form:$r_i^2=\gamma_0+\gamma_1\hat{y}_i+\gamma_2\hat{y}_i^2+e_i$
  
  \textbf{Normality}: \underline{Shapiro-Wilk test} and \underline{Kolmogorov-Smirnov test}\\
  \underline{SW}: test statistic: $W=\frac{(\sum_{i=1}^{n}a_ir_{(i)} )^2}{\sum_{i=1}^n(r_i-\overline{r})^2 }$, $0\leq W\leq 1$ and small values of W lead to rejection of normality. \\
  Dist of W under norm has no closed form, only applied when $n\leq$2000.\\
  \underline{KS}: based on empirical (edf), $F_n(x)=\frac{1}{n}\sum_{i=1}^{n}I_{(r_i\leq x)} $, test statistic:
  $D=sup_x|F_n(x)-F(X)|$, F is normal cdf, $D\sim_{asy}$Kolmogorov dist under normality,  K-S test requires a relatively large$(n>2k)$ to take proper cdf\\
  \textbf{Independence}: \underline{Durbin-Watson Test}, we can judge whether it is reasonable to assume independence based on the nature of how the data were collected. \\
  for time series data,
  test statistic: $DW=\frac{\sum_{t=2}^n(\hat{\epsilon}_{t}-\hat{\epsilon}_{t-1})^2}{\sum_{t=1}^n \hat{\epsilon}_t^2 }$, detect the first order auto-corr(ass $\epsilon_t=\rho \epsilon_{t-1}+u_t$), $H_0$: $\rho=0$. $1 \leq DW \leq 4$, and DW=2 indicates no auto-corr, DW$<1$ means strong postive auto-corr, DW$>3$ means strong negative auto-corr.
  
  \textbf{5.3 Unusual Observation}\\
  


	
	
	
	
  {\small\textbf{6 Analysis of Variance}}\\
  \textbf{6.1.1 Definition and F-test}\\
  ANOVA is used to analyze the differences among group means in a sample\\
  The model is $Y_{ij}\sim N(\mu_i,\sigma^2 )$,$i=1,2,\cdots,k$.\\
  \textbf{means model}:$Y_{ij}=\mu_i+\epsilon_{ij}$and \textbf{effect model}:$Y_{ij}=\mu+\alpha_i+\epsilon_{ij}$ \\
  where $\epsilon_{ij}\sim_{iid}N(0,\sigma^2) $ is random error, $\alpha_i=\mu_i-\mu$:main effect of group i.
  Hypothesis: $H_0:\mu_1=\mu_2=\cdots = \mu_k$ vs. $H_1:$ not all $\mu_i$ are equal.\\
  Assumptions: Normality \& Homoscedasticity \& Independence \\
  SSB=$\sum_{i=1}^{k}\sum_{j=1}^{n_i}(\overline{Y_i}-\overline{Y} )^2=\sum_{i=1}^kn_i(\overline{Y}_i-\overline{Y} )^2 $,variation between groups \\
  SSW=$\sum_{i=1}^{k}\sum_{j=1}^{n_i}(\overline{Y}_{ij}-\overline{Y}_i )^2=\sum_{i=1}^k(n_i-1)S_i^2 $,variation within groups\\
  Under assumptions of independence and equal variance, we have $E(SSB)=(k-1)\sigma^2+\sum_{i=1}^k n_i(\mu_i-\mu)^2 $ and $E(SSW)=(n-k)\sigma^2$\\
  The test statisitc: $F=\frac{SSB/(k-1)}{SSW/(n-k)}$\\
  SSB$\bot$SSW under $H_0$, and SSB+SSW=SST=$\sum_{i=1}^k\sum_{j=1}^{n_i}(Y_{ij}-\overline{Y}) ^2$\\
  With normality \& under $H_0$:$\frac{SST}{\sigma^2}\sim \chi^2_{n-1} ,\frac{SSB}{\sigma^2}\sim \chi^2_{k-1} ,\frac{SSW}{\sigma^2}\sim \chi^2_{n-k} $\\
  Therefore, under $H_0:F=\frac{SSB/(k-1)}{SSW/(n-k)}\sim F(k-1,n-k)$ (one-side test)
  \begin{table}[H]
  	\scriptsize
  	\begin{tabular}{c|cccc}
  		\hline
  		Source & df           & SS   & MS           & F Value                           \\
  		\hline
  		Between     & $k-1$        & SSB  & ${\rm MSB}$  & $F=\frac{\rm MSB}{\rm MSW}$   \\
  		Within     & $n-k$        & SSW  & ${\rm MSW}$  &     \\
  		Total  & $n-1$        & SST  &              &             \\
  	\end{tabular}
  \end{table}
  \textbf{6.1.2 Testing Equality of Group Variance(homoscedasticity)}\\
  Hypothesis: $H_0:\sigma^2_1=\sigma^2_2=\cdots=\sigma^2_k$ vs. $H_1:$ Not all $\sigma^2_i$ are equal.\\
  \underline{Bartlett's}: For $a_1,a_2,\cdots,a_k>0$, the weighted arithmetic mean and geometric mean are $ \overline{a}^A=\sum_{i=1}^k w_ia_i, \overline{a}^G=\Pi_{i=1}^k a_i^{w_i}$. \\
  For $\sum_{i=1}^kw_i=1$, $\overline{a}^A>\overline{a}^G$ and attain equality iff $a_1,a_2,\cdots,a_k$ are all equal. Let $w_i=\frac{n_i-1}{n-k}$, then $\overline{S}^2_A=\sum_{i=1}^k w_iS_i^2,\overline{S}^2_G=\Pi_{i=1}^k (S_i^2)^{w_i} $.\\
  Test statistic:$B=\frac{(n-k)(log\overline{S}^2_A-log\overline{S}^2_G) }{1+\frac{1}{3(k-1)}[(\sum_{i=1}^k\frac{1}{n_i-1} )-\frac{1}{n-k}]}\sim_{approx} \chi^2_{k-1} $ under $H_0$ (approx need normality\&large sample)\\
  Also$\frac{(n-k)InS^2-\sum_{i=1}^k(n_i-1)S_i^2 }{1+\frac{1}{3(k-1)}[(\sum_{i=1}^k\frac{1}{n_i-1} )-\frac{1}{n-k}]} $ where $S^2=\sum_{i=1}^k(n_i-1)S_i^2/(n-k)$\\
  \underline{Levene's and Brown-Forsythe test}:
  Transform the original values of $Y_{ij}$ to dispersion variable $Z_{ij}$, perform ANOVA on $Z_{ij}$.\\
  Levene's: $Z_{ij}=(Y_{ij}-\overline{Y}_i)^2$or$|Y_{ij}-\overline{Y}_i|$ (these two tests are robust)\\
  BF use $Z_{ij}=|Y_{ij}-m_i|$,$m_i$ is sample median of group $i$.\\
  
  \textbf{6.1.3 Kruskal-Wallis Test (Nonparameter test)}\\
  When normality is violated, nonparametric alternative to one-way ANOVA, relpacing $y_{ij}$ by rank$\rightarrow$test the equality of population group median.\\
  Rank all $Y_{ij}$'s from all groups together, denoted $R_{ij}$, $\overline{R}_i=\sum_{j=1}^{n_i}R_{ij}/n$ and $\overline{R}=(n+1)/2$\\
  Test statistic: $KW=\frac{(n-1)\sum_{i=1}^k n_i(\overline{R}_i-\overline{R})^2 }{\sum_{i=1}^k\sum_{j=1}^{n_i}(R_{ij}-\overline{R})^2}\sim _{approx.}\chi^2_{k-1}$under $H_0$.\\
  $H_0$ can be rejected if $KW>\chi^2_{\alpha,k-1}$.\\
  \underline{Multiple Comparsions}: if the CI of $\mu_i-\mu_j$ contains 0, $\mu_i$ and $\mu_j$ are not significantly different. where \underline{Tukry-Kramer} for comparsions bet all pairs of means and \underline{Dunnett} for comparsions bet a control and all other means.
  
  
  
   \textbf{6.2 Two-way ANOVA}\\
   Models: means: $Y_{ijk}=\mu_{ij}+\epsilon_{ijk}$\&effects:$Y_{ijk}=\mu+\alpha_i+\beta_i+\gamma_{ij}+\epsilon_{ijk}$\\
   $\alpha_i$,$\beta_i$ are the main effect of factor, $\gamma_{ij}$ is the interaction effect bet A and B.\\
   Interaction effect means the effect of one factor depends on the level of the other factor.\\
   For inreraction model: $\alpha_i=\mu_{i\cdot}-\mu,\beta_j=\mu_{\cdot j}-\mu$\\
   then $\gamma_{ij}=\mu_{ij}-(\mu+\alpha_i+\beta_j)=\mu_{ij}-\mu_{i \cdot}-\mu_{\cdot j}+\mu $\\
   The SSE to be minimized is $SSE=\sum_{i=1}^a\sum_{b=1}^b\sum_{k=1}^{n_{ij}}(Y_{ijk}-\mu_{ij})^2 $\\
   LSE estimator: $\hat{\mu_{ij}}=\overline{Y}_{ij} , \hat{\mu}=\overline{Y}, \hat{\alpha_i}=\overline{Y}_{i\cdot}-\overline{Y}, \hat{\beta_i}=\overline{Y}_{\cdot j}-\overline{Y}$, and
   $\hat{\gamma}_{ij}=\overline{Y}_{ij}-\overline{Y}_{i\cdot}-\overline{Y}_{\cdot j}+\overline{Y}$\\  
   \underline{Test interaction effect}: $H_0^{AB}:\gamma_{ij}=0$for$i=1,\cdots,a,j=1,\cdots,b$\\
   If $H_0^{AB}$ is not reject, then test the main effect of each factor\\
   $SSM=\sum_{i=1}^a\sum_{j=1}^b n_{ij}(\overline{Y}_{ij}-\overline{Y})^2$,$SSW=\sum_{i=1}^a\sum_{j=1}^b \sum_{k=1}^{n_{ij}}(Y_{ijk}-\overline{Y}_{ij})^2$\\
   $SST=\sum_{i=1}^a\sum_{j=1}^b \sum_{k=1}^{n_{ij}}(Y_{ijk}-\overline{Y})^2$ and SST=SSM+SSE\\
   F-test statisitic:$F=\frac{SSM/(ab-1)}{SSE/(n-ab)} $ to $H_0$:interaction model vs.$H_1$:null model\\
   furtherly decompose the variation between groups  to difference sources.\\
   Consider $\overline{Y}_{ij}-\overline{Y}=(\overline{Y}_{i\cdot}-\overline{Y})+(\overline{Y}_{\cdot j}-\overline{Y})+(\overline{Y}_{ij}-\overline{Y}_{i\cdot}-\overline{Y}_{\cdot j}+\overline{Y})$\\
   Define $SSA=\sum_{i=1}^a n_{i\cdot}(\overline{Y}_{i\cdot}-\overline{Y})^2$ (variation between groups due to factor A), define $SSB=\sum_{j=1}^b n_{\cdot j}(\overline{Y}_{\cdot j}-\overline{Y})^2$, define $SSAB=\sum_{i=1}^a\sum_{j=1}^b n_{ij}(\overline{Y}_{ij}-\overline{Y}_{i\cdot}-\overline{Y}_{\cdot j}+\overline{Y} )^2$(variation between groups due to the interaction of factor A and B)\\
   Under ass of indep\&homoscedasticity, $E(SSA)=(a-1)\sigma^2+\sum_{i=1}^a n_{i\cdot}\alpha_i^2$, $E(SSB)=(b-1)\sigma^2+\sum_{i=1}^a n_{\cdot j}\beta_j^2$, $E(SSAB)=(a-1)(b-1)\sigma^2+\sum_{i=1}^a\sum_{j=1}^b n_{ij}\gamma_{ij}^2$\\
   F-test: $H_0^{AB}$:All $\gamma_{ij}=0$, $H_0^A$:All $\alpha_i=0$, $H_0^B$:All $\beta_j=0$
   \begin{table}[H]
	\scriptsize
	\begin{tabular}{c|cccc}
		\hline
		Source & df           & SS   & MS           & F Value                           \\
		\hline
		A      & $a-1$        & SSA  & ${\rm MSA}$  & $F^A=\frac{\rm MSA}{\rm MSE}$     \\
		B      & $b-1$        & SSB  & ${\rm MSB}$  & $F^B=\frac{\rm MSB}{\rm MSE}$     \\
		A*B    & $(a-1)(b-1)$ & SSAB & ${\rm MSAB}$ & $F^{AB}=\frac{\rm MSAB}{\rm MSE}$ \\
		Error  & $n-ab$       & SSE  & ${\rm MSE}$  &                                   \\
		Total  & $n-1$        & SST  &              &                                   \\
	\end{tabular}
\end{table}
   However, SSM=SSA+SSB+SSAB is only true when all $n_{ij}$ are equal.\\
   \textbf{6.2.3 Type I and Type III SS}\\
   We can consider the SS for a given source to be the extra variability explained when the respective term is added to  the model, i.e., reduction in when the term is added.\\
   SSA=SSE(null)-SSE(A), SSB=SSE(A,B)-SSE(A,B,AB), SSAB=SSE(A,B)-SSE(A,B,AB)\\
   \underline{The order in which terms are entered into the model matters.}\\
   
   Difference:In Type I, effects are added sequentially. In Type III, assumed that all the effects are already in the model other than the effect of interest.\\
   SSA=SSE(B,AB)-SSE(A,B,AB),SSAB=SSE(A,B)-SSE(A,B,AB)\\
   In a balanced design, Type I and Type III SS are the same, because each effect provides unique
   information and doesn’t take away from what another effect explains.\\
   The order in which terms are entered into the model does not change the Type III SS, not satisfy SSM=SSA+SSB+SSAB
   
   
   
   
   {\small\textbf{7 Generalized Linear Models}}\\
   \textbf{7.1 Exponential Family and Generalized Linear Models}\\
   A pmf/pdf belongs to exponential family of distributions if it is of the form: $f(\boldsymbol{y};\boldsymbol{\theta})=h(\boldsymbol{y})exp\{\boldsymbol{\eta}(\boldsymbol{\theta})\cdot \boldsymbol{T}(\boldsymbol{y})-A(\boldsymbol{\theta}) \}$\\
   If $\boldsymbol{\eta}(\boldsymbol{\theta})=\boldsymbol{\theta}$, called canonical form(where $\theta$ called canonical parameter)\\
   T($\boldsymbol{y}$) is the sufficient statistic of the natural parameter $\boldsymbol{\theta}$\\
   If $T(\boldsymbol{y})=\boldsymbol{y}$, then the family is called a natural exponential family.\\
   Binomial, poisson, exponential, normal belongs to exponential family.\\
   Property of exp\_fam: for canonical form:$f(\boldsymbol{y};\boldsymbol{\theta})=h(\boldsymbol{y})exp\{\boldsymbol{\theta}\cdot \boldsymbol{T}(\boldsymbol{y})-A(\boldsymbol{\theta}) \}$, (1) mgf of $T(\boldsymbol{Y})$ is $M_T(\boldsymbol{t})=exp\{A(\boldsymbol{t+\theta})-A(\boldsymbol{\theta}) \}$,(2)$E(T(\boldsymbol{Y}))=A'(\theta), Var[T(\boldsymbol{Y})]=A''(\theta)$\\
   if consider the $\phi$,we have $E(T(\boldsymbol{Y}))=A'(\theta), Var[T(\boldsymbol{Y})]=\phi A''(\theta)$\\
   link function g s.t. $g(\mu_i)=\beta_0+\beta_1x_{i1}+\cdots+\beta_px_{ip}=\boldsymbol{x}_i^T \boldsymbol{\beta}=\eta_i$ linear predictor.\\
   canconical link:$g(\cdot)=(A')^{-1}(\cdot)$s.t.$g(E(Y_i))=(A')^{-1}(A'(\theta_i))=\theta=\eta_i$\\
   \textbf{parameter estimation:} $l(\boldsymbol{\beta})=\sum_{i=1}^n[\theta_iy_i-A(\theta_i)]$\\
   if use canonical link: $\frac{\partial l}{\partial \boldsymbol{\beta}}=\sum_{i=1}^n \boldsymbol{x_i}(y_i-\mu_i)=\sum_{i=1}^n \boldsymbol{x_i}(y_i-g^{-1}(\boldsymbol{x_i^T\beta})) $\\
   general link: $\frac{\partial l}{\partial \boldsymbol{\beta}}=\sum_{i=1}^n \frac{\boldsymbol{x}_i (y_i-\mu_i)}{Var(Y_i)g'(\mu_i)} $ \& score function $\boldsymbol{U(\beta)}=\frac{\partial l}{\partial \boldsymbol{\beta}}$ \\
   iterative alg: $\boldsymbol{\beta}^{(m+1)}=\boldsymbol{\beta}^{(m)}+[\boldsymbol{J(\beta^{(m)})}]^{-1}\boldsymbol{U(\beta^{(m)})}$ where $\boldsymbol{J(B)}$ is usually replaced by $\boldsymbol{I(\beta)}=E[\boldsymbol{J(\beta)}]$ Fisher inofrmation matrix\\
   Another Iteratively Reweighted Least Squares(IRLS):
   $\boldsymbol{\beta}^{(m+1)}=(\boldsymbol{X^TW^{(m)}X})^{-1}\boldsymbol{X^TW^{(m)}z^{(m)}} $ with $\boldsymbol{W}^{(m)}=diag\{w_i^{(m)} \}$\\
   where $z_i^{(m)}=\eta_i^{(m)}+(y_i-\mu_i^{(m)})g'(\mu_i^{(m)})$(working response) and $w_i^{(m)}=\frac{1}{Var(Y_i|\beta^{(m)})[g'(\mu_i^{(m)})]^2} $(working weight matrix)\\
   \textbf{CI:} score statistic: $\boldsymbol{U}=\frac{\partial l}{\partial \boldsymbol{\beta}} =\sum_{i=1}^n \frac{\boldsymbol{x}_i, (y_i-\mu_i)}{Var(Y_i)g'(\mu_i)} $  we have $E(\boldsymbol{U})=0$, variance matrix of $\boldsymbol{U}$ is $\boldsymbol{V}=E(\boldsymbol{UU^T})$with (j,k) element: $v_{jk}=\sum_{i=1}^n \frac{x_{ij}x_{ik}}{Var(Y_i)[g'(\mu_i)]^2} $ rewrite as: $\boldsymbol{V}=\boldsymbol{X^TWX}$, where $\boldsymbol{W}$ is an n by n matrix with elements $w_{ii}=\frac{1}{Var(Y_i)[g'(\mu_i)^2]}$.\\
   $\boldsymbol{U}\sim N(\boldsymbol{0,V})$ (asymptotically), $\boldsymbol{\hat{\beta}}\sim N(\boldsymbol{\beta, V^{-1}})$(asymptotically)\\
   
   
   
   \textbf{Goodness of fit}
   Deviance: $D(M)=2(l(\hat{\boldsymbol{\beta}_S})-l(\hat{\boldsymbol{\beta}_M} ))$ where $g^{-1}(\hat{\beta}^S_i)=\hat{\mu}_i^S=y_i $, note that $D^*=\frac{D}{\phi} \sim \chi^2_{n-p-1}$ (asy), deviance residual: $r_{Di}=sign(y_i-\hat{\mu}_i)\sqrt{d_i}$,$d_i$ is the contribution of the ith obs to the deviance\\
   Generalized Pearson's Chi-Square statistic; $\chi^2=\sum_{i=1}^n\frac{(y_i-\hat{\mu}_i)^2}{V(\hat{\mu}_i)} $, $r_{Pi}=\frac{(y_i-\hat{\mu}_i)}{\sqrt{V(\hat{\mu}_i)}}$, scaled version $\chi^{2*}=\frac{\chi^2}{\phi}\sim_{asy}\chi^2_{n-p-1} $\\
   \textbf{Logistic regression}: why logit link: 1.canonical link 2.to $\infty$ 3.good interpretation.
   odds=$\frac{p}{1-p}$, An event with odd$>1$ is more likely to happen than not happen. OR=$\frac{odds_1}{odds_2}$, OR$>1$ indicates that the event is more likely to happen in the first population.\\
   When $X_j$is a continuous explanatory variable, with all other $x_l$'s fixed, if $x_j$ increases by 1 unit, the odds of Y=1 changes by a multiplicative factor of exp($\beta_j$).\\
   A worker under...has 14.42 times the odds of...compared to
   
   
   
   

}  
  





\end{multicols}

\end{document}